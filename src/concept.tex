\section{Upgrade objectives, motivations and goals (10 p.)}

The ALICE detector has undergone a major upgrade during the Long Shutdown 2 of the LHC, 2019-2021.
The main goal of the upgrade is to improve the performance for rare, untriggered signals: the production of open heavy flavor mesons and baryons which serve as important 'trace particles' that probe both early and late time dynamics in the collision, as well as electron-positron pairs which measure the temperature of the hot and dense Quark Gluon Plasma produced in the collision, as well as effects related to chiral symmetry restoration.
To achieve this, the full Inner Tracking System has been replaced with a new system based on monolithic active pixel sensors with a better spatial resolution than the previous ITS, as well as a lower material budget, to further improve the spatial resolution and reduce the background from photon conversions. The readout chambers of the TPC have been replaced with multi-foil GEM chambers to reduce the ion backflow and allow continuous readout of the TPC.
In the forward direction, the new Muon Forward Tracker (MFT) is a silicon pixel tracker that provides accurate pointing for muon tracks, to identify quarkonia from beauty decays.
The trigger and data acquisition systems have been completely redesigned and replaced with an online processing system that performs important parts of the reconstruction, to reduce the data rate while keeping information on all events and primary tracks. The readout electronics of several sub-detectors has been replaced to provide larger readout rates than were previously possible and to be compatible with the new online data acquisition and processing system. 

\subsection{Physics performance (Marco, Jochen)}

physics goals for Run 3/4, refer to yellow report and TDRs~\cite{Citron:2018lsq}

significant increase of interaction rate needed (not possible with gated TPC)

rare probes, which cannot be triggered on (low pt, no unique signature)

maintaining PID capabilities

\subsection{System design (Werner, Alex)}
\begin{itemize}
\item Radiation tolerance/load
\item continuous read-out, no filtering but compression (general description), TF and its related impact on reconstruction, CTP-CRU-FLP-EPN approach, implementation specification (beam, particle load, radiation load)
\end{itemize}
