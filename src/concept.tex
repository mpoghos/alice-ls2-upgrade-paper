\section{Upgrade objectives, motivations and goals (10 p.)}

The ALICE detector has undergone a major upgrade during the Long Shutdown 2 of the LHC, 2019-2021.
The main goal of the upgrade is to improve the performance for rare, untriggered signals: the production of open heavy flavor mesons and baryons which serve as important 'trace particles' that probe both early and late time dynamics in the collision, as well as electron-positron pairs which measure the temperature of the hot and dense Quark Gluon Plasma produced in the collision, as well as effects related to chiral symmetry restoration.
To achieve this, the full Inner Tracking System has been replaced with a new system based on monolithic active pixel sensors with a better spatial resolution than the previous ITS, as well as a lower material budget, to further improve the spatial resolution and reduce the background from photon conversions. The readout chambers of the TPC have been replaced with multi-foil GEM chambers to reduce the ion backflow and allow continuous readout of the TPC.
In the forward direction, the new Muon Forward Tracker (MFT) is a silicon pixel tracker that provides accurate pointing for muon tracks, to identify quarkonia from beauty decays.
The trigger and data acquisition systems have been completely redesigned and replaced with an online processing system that performs important parts of the reconstruction, to reduce the data rate while keeping information on all events and primary tracks. The readout electronics of several sub-detectors has been replaced to provide larger readout rates than were previously possible and to be compatible with the new online data acquisition and processing system. 

\subsection{Physics performance (Marco, Jochen)}

physics goals for Run 3/4, refer to yellow report and TDRs~\cite{Citron:2018lsq}

\begin{itemize}
\item precision measurement of long-wavelength behaviour\\
 material properties and phase transition
\item access microscopic parton dynamics (underlying QGP)\\
 degrees of freedom and interactions
\item unify picture of particle production across systems\\
 collectivity and validity of fluid description
\item constrain nuclear parton densities over a wide (x, Q2) range\\
 initial stages
\end{itemize}

\subsubsection{Heavy-flavour and quarkonia production}

\subsubsection{Low-mass dileptons}

\subsubsection{Jets}

\subsubsection{Nuclear states}

hyper nuclei

\subsubsection{Small systems}

particle production and energy loss

\subsubsection{small $x$}

nuclear PDFs

\subsubsection{Requirements}

These physics goals translate to requirements on operational conditions and detector systems. While the details shall be discussed in the next section, we will only summarise the essence here. The high interaction rates needed to accumulate enough integrated luminosity (up to 50 kHz \PbPb{} and up to 1 MHz \pp{}), require the operation of the TPC without gating and, thus, with significantly reduced ion back flow. The precise reconstruction of secondary vertices can be achieved by an Inner Tracking System, with a smaller radius of the innermost layer and with a better position resolution in each layer. Probes which only show as a small signal on top of large combinatorial background, cannot be triggered and require large statistics to be analysed. This implies the processing of un-preselected (untriggered) events from the continuosly read-out TPC. In addition, the performance of particle identification must not be deteriorated in order to extract significant measurements in environments of large combinatorial background.
\begin{itemize}
\item high interaction 
\end{itemize}

\subsection{System design (Werner, Alex)}
\begin{itemize}
\item Radiation tolerance/load
\item continuous read-out, no filtering but compression (general description), TF and its related impact on reconstruction, CTP-CRU-FLP-EPN approach, implementation specification (beam, particle load, radiation load)
\end{itemize}
